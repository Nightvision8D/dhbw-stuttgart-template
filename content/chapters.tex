%!TEX root = ../main.tex

\input{content/10_Problemstellung}

%!TEX root = ../main.tex

\chapter{Grundlagen}
Test von der Aufgabenstellung ist der derzeitige Stand der Technik für die Lösungsfindung zu beschreiben. Es sind z.B. die Vor- und Nachteile bisheriger Lösungen bzw. fundamentaler Lösungsprinzipien fundiert von und ggf. anderen Quellen darzulegen.

\input{content/30_Hauptteil}
%!TEX root = ../main.tex

\chapter{Zusammenfassung}
Aufgabenstellung, Vorgehensweise und wesentliche Ergebnisse werden kurz und präzise dargestellt und kritisch reflektiert. Die Zusammenfassung ist eigenständig verständlich. Länge ca. 1 bis 1,5 Seiten (Problem, Ziele, Vorgehensweise, Ergebnisse und Ausblick).

\chapter{Ausblick}
Test obs funktioniert


%% Beispiele, können problemlos entfernt werden!
\chapter{Beispiele}
Beispiele, können problemlos entfernt werden!
\section{Literatur} 
 
\section{Bilder}
\begin{figure}[H]
	\centering
	\includegraphics[width=0.3\linewidth]{resources/images/logo-dhbw}
	\caption{DHBW Logo}
	\label{fig:logo-dhbw}
\end{figure}

\section{Fußnote und Abkürzung}
Fußnote\footnote{Fußnote}, \ac{DHBW}

\section{Tabelle}
\begin{table}[H]
	\centering
	\caption{Tabellenbeispiel}
	\label{tab:example}
	\begin{tabular}{|l|c|r|}
		\hline
		Spalte 1 & Spalte 2 & Spalte 3 \\
		\hline
		Zeile  &  &  \\
		\hline
		& Zeile &  \\
		\hline
		&  & Zeile \\
		\hline
		\multicolumn{2}{|r|}{Verbunden}	& nicht Verbunden \\
		\hline
	\end{tabular}
\end{table}

\section{Skript}
\lstinline[language=c]|printf("Inline Code");|
\lstinputlisting[language=python,caption={Beispiel python Skript},captionpos=t,label=scr:exanple]{resources/example-script.py}

\section{TODOs}
Text Text\todo{TODO als Randnotiz} Text
\todo[inline]{TODO als in Zeile}

\begin{landscape}
	\section{Querformat}
	Text über Tabelle \ref{tab:breitetabelle}.
	\begin{table}[H]
		\caption{Breites Tabellenbeispiel}
		\label{tab:breitetabelle}
		\centering
		\resizebox{\columnwidth}{!}{%
			\begin{tabular}{|p{10cm}|p{10cm}|p{10cm}|}
				\hline
				Sehr & breite & Tabelle \\
				\hline
				\multicolumn{3}{|c|}{Tabelle wurde in der Größe verkleinert} \\
				\hline
			\end{tabular}
		}
	\end{table}
	Text unter Tabelle \ref{tab:breitetabelle}.
\end{landscape}

\section{Auflistungen}
\begin{enumerate}
	\item Aufzählung nummeriert
\end{enumerate}
\begin{itemize}
	\item Aufzählung Stichpunkte
\end{itemize}
\begin{description}[style=nextline]
	\item[Label] Aufzählung als Beschreibung
\end{description}
